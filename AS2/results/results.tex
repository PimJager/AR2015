\documentclass[12pt]{article}
\usepackage{a4wide}
\usepackage{latexsym}
\usepackage{amssymb}
\usepackage{epic}
\usepackage{graphicx}
\usepackage{amsmath}
\usepackage{enumerate}
%\pagestyle{empty}
\newcommand{\tr}{\mbox{\sf true}}
\newcommand{\fa}{\mbox{\sf false}}
\newcommand{\bimp}{\leftrightarrow}


\begin{document}

\section*{Automated Reasoning - Assignment series 2 }

\begin{center}
P.T. Jager, BSc. \\
Radboud universiteit Nijmegen\\
email: {\tt p.jager@student.ru.nl}
\end{center}

\subsection*{Problem 2: Bottles}

Three bottles can hold 144, 72 and 16 units (say, centiliters), respectively.
Initially the first one contains 3 units of water, the others are empty. 
The following actions may be performed any number of times:

\begin{itemize}
	\item One of the bottles is fully filled, at some water tap.
	\item One of the bottles is emptied.
	\item The content of one bottle is poured into another one. If it fits, 
			then the full content is poured, otherwise the pouring stops when 
			the other bottle is full.
\end{itemize}

\begin{enumerate}[(a)]
	\item Determine whether it is possible to arrive at a situation in which 
	the first bottle contains 8 units and the second one contains 11 units. If 
	so, give a scenario reaching this situation.
	\item Do the same for the variant in which the second bottle is replaced by 
	a bottle that can hold 80 units, and all the rest remains the same.
	\item Do the same for the variant in which the third bottle is replaced by a
	bottle that can hold 28 units, and all the rest (including the capacity of 
	72 of the second bottle) remains the same.
\end{enumerate}

\subsubsection*{model}
The problem can be modeled in Yices in the following way:

We introduce $3*m$ variables, one for the contents of each bottle in timestep 
$n$ where $m$ is the total number of timesteps to consider.

\begin{itemize}
\item $c\_x\_n$ for $x \in B$ and $n = 1 \ldots m$
\end{itemize}

We also assume a function $max(x)$ which yields the maximum contents of bottle
$x$, i.e. $max(1) = 144$. $B$ will be the set of all bottles, i.e. $\{1,2,3\}$.

The following formulas describe the constraints of the model.

\vspace{3mm}

All bottles should contain at least no water, and at max their maximum capacity
at any given timestep.
\begin{equation}
	\bigwedge_{x \in B} \bigwedge_{n \in \{1\ldots m\}} 
							0 \leq c\_x\_n \leq max(x)
\end{equation}
 
\vspace{3mm}

All bottles should contain the correct amount of water in the first timestep:
\begin{equation}
	c\_1\_1 = 3 \wedge c\_2\_1 = 0 \wedge c\_3\_1 = 0
\end{equation}

\vspace{3mm}

The model satisfies the condition if at any given timestep the bottles contain
the required contents:
\begin{equation}
	\bigvee_{n \in \{2\ldots m\}} c\_1\_n = 8 \wedge c\_2\_n = 11
\end{equation}

This model can trivially be extended for $n$ bottles. 

\subsubsection*{Discussion}
We have constructed two different models, one in Yices and one for NuSMV. 

NuSMV might seem to be a better tool to model the problem at first hand. However
do to poor experiences with the tool in the past the problem is modeled for 
YICES. This model is fast enough for assignment (a).

\end{document}
