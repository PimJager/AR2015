\documentclass[12pt]{article}
\usepackage{a4wide}
\usepackage{latexsym}
\usepackage{amssymb}
\usepackage{epic}
\usepackage{graphicx}
\usepackage{amsmath}
\usepackage{enumerate}
\usepackage{listings}
\usepackage{float}
\usepackage{multirow}
%\pagestyle{empty}
\newcommand{\tr}{\mbox{\sf true}}
\newcommand{\fa}{\mbox{\sf false}}
\newcommand{\bimp}{\leftrightarrow}

\lstset{ keepspaces=true
        ,captionpos=b
        ,basicstyle=\scriptsize\ttfamily
        ,escapechar=§}

\begin{document}

\section*{Automated Reasoning - Assignment series 2 }

\begin{center}
P.T. Jager, BSc. \\
Radboud universiteit Nijmegen\\
email: {\tt p.jager@student.ru.nl}
\end{center}

\subsection*{Problem 2: Bottles}

Three bottles can hold 144, 72 and 16 units (say, centiliters), respectively.
Initially the first one contains 3 units of water, the others are empty. 
The following actions may be performed any number of times:

\begin{itemize}
    \item One of the bottles is fully filled, at some water tap.
    \item One of the bottles is emptied.
    \item The content of one bottle is poured into another one. If it fits, 
            then the full content is poured, otherwise the pouring stops when 
            the other bottle is full.
\end{itemize}

\begin{enumerate}[(a)]
    \item Determine whether it is possible to arrive at a situation in which 
    the first bottle contains 8 units and the second one contains 11 units. If 
    so, give a scenario reaching this situation.
    \item Do the same for the variant in which the second bottle is replaced by 
    a bottle that can hold 80 units, and all the rest remains the same.
    \item Do the same for the variant in which the third bottle is replaced by a
    bottle that can hold 28 units, and all the rest (including the capacity of 
    72 of the second bottle) remains the same.
\end{enumerate}

\subsubsection*{Model}
The problem can trivially be modeled as LTL for NuSMV. The equations below are 
written in a quantified pseudo language which resembles NuSMVs syntax and by 
expanding the quantifiers can be expanded to a NuSMV program. 

\vspace{1mm}

The amount of water in each bottle is modeled using a variable $c_{xn}$ where 
$x$ is the number of the bottle and $n$ counts the timesteps. The value of 
$c_{xn}$ ranges 
from the bottle being empty (0) to the bottle begin at max capacity, for this we
assume a function $max(x)$ which yields the maximum capacity of the bottle 
$x$, i.e. $max(1) = 144$.

And for all these contents the initial value has to be set, for this we assume
a function $init(x)$ which yields the initial contents of the bottle $x$, i.e. 
$init(1) = 3$.
\begin{align*}
        &\forall_{x \in \{1,2,3\}} c_{xn} : 0 \ldots max(x) \\  
        &\bigwedge_{x \in \{1,2,3\}} c_{x0} = init(x)
\end{align*}

\vspace{3mm}

The following transition function defines that at each timestep one of the three
actions described above could be done to one of the three bottles. (2) 
describes the action of emptying or filling bottle $c_x$ and not touching the 
other two bottles. (5) to (8) describe pouring the contents of bottle $c_y$
 into 
$c_x$ if it fits completely (6) or partially (7) and not touching the remaining
bottle $c_z$.
\begin{align}
    \bigvee_{x \in \{1,2,3\}} &  \\  
            &\left( 
                (c_{x(n+1)} = max(x) \vee c_{x(n+1)} = 0) 
                    \wedge 
                    \left(\bigwedge_{y \in \{1,2,3\} \setminus \{x\}} 
                        c_{y(n+1)} = c_{yn}\right)
             \right) \\  
             &\vee \\  
             &\bigvee_{y \in \{1,2,3\} \setminus \{x\}} 
                \Big(\\  
                    &case\\  
                    &c_{xn} + c_{yn} \leq max(x) \to 
                        c_{x(n+1)} = (c_{xn} + c_{yn}) \wedge c_{y(n+1)} = 0 
                            \wedge \bigwedge_{z \in \{1,2,3\} \setminus \{x,y\}}
                                c_{z(n+1)} = c_{zn}\\  
                    &otherwise  \to
                        c_{x(n+1)} = max(x) \wedge 
                        c_{y(n+1)} = c_{yn} - (max(x) - c_{xn}) \wedge 
                        \bigwedge_{z \in \{1,2,3\} \setminus \{x,y\}}
                            c_{z(n+1)} = c_{zn}\\  
                    &esac\\  
                \Big)
\end{align}

\vspace{3mm}

To find if it is possible to reach the desired bottle contents we ask NuSMV to 
prove that the this desired state is never reached. That is, the following 
formula should hold for every n (globally).
\begin{equation} \label{eq:ltlspec}
    \neg (c_{1n} = 8 \wedge c_{2n} = 11)
\end{equation}
If NuSMV finds a sequence that violates this formula, and thus reaches our
desired state, it will show this sequence. 

\subsubsection*{solution (a)}
If we specify the $max(x)$ as follows:
\begin{equation*}
max(x) = 
    \begin{cases}
        144 & x = 1\\  
        72  & x = 2\\  
        16  & x = 3\\  
    \end{cases}
\end{equation*}
NuSMV indeed finds a sequence that shows that it is possible to have 8 units in 
bottle 1 and 11 units in bottle 2 in under second. This sequence is shown
in the table below.

\begin{tabular}{|l|l|l|l|}
    \hline  
    n & bottle 1    & bottle 2  & bottle 3  \\  
    \hline \hline
    1 & 3   & 0     & 0 \\  
    2 &     & 72    &   \\  
    3 & 75  & 0     &   \\      
    4 & 59  &       & 16 \\  
    5 &     & 16    & 0 \\  
    6 & 43  &       & 16 \\  
    7 &     & 32    &   \\  
    8 & 27  &       &   \\  
    9 &     & 48    & 0 \\  
    10&     &       & 16 \\  
    11&     & 64    & 0 \\  
    12& 11  &       & 16 \\  
    13&     & 72    & 8 \\  
    14&     & 0     &   \\  
    15& 0   & 11    &   \\  
    16& 8   &       & 0 \\  
    \hline
      & 8   & 11    & 0 \\  
    \hline
\end{tabular}

\subsubsection*{Solution (b)}
Similar to (a) we define the $max(x)$ function as follows:
\begin{equation*}
max(x) = 
    \begin{cases}
        144 & x = 1\\  
        80  & x = 2\\  
        16  & x = 3\\  
    \end{cases}
\end{equation*}
NuSMV determines in under a seconds that our specification is true. That is, 
formula~\ref{eq:ltlspec} holds for every $n$ and it is not possible to have the 
desired amounts in bottle 1 and 2.

\subsubsection*{Solution (c)}
Similar to (a) we define the $max(x)$ function as follows:
\begin{equation*}
max(x) = 
    \begin{cases}
        144 & x = 1\\  
        72  & x = 2\\  
        28  & x = 3\\  
    \end{cases}
\end{equation*}
NuSMV finds a sequence that shows our specification is false, thus that is it is
possible to reach the desired state, in just over a second. The table below 
shows this sequence.

\begin{tabular}{|l|l|l|l|}
    \hline  
    n & bottle 1    & bottle 2  & bottle 3  \\  
    \hline \hline
    1 & 3   & 0     & 0 \\  
    2 & 0   & 3     &   \\  
    3 & 144 &       &    \\  
    4 & 116 &       & 28 \\  
    5 &     & 31    & 0 \\  
    6 & 88  &       & 28 \\  
    7 &     & 59    & 0 \\  
    8 & 60  &       & 28 \\  
    9 &     & 72    & 15 \\  
    10& 143 &       & 0 \\  
    11&     & 15    & 0 \\  
    12& 104 &       & 28 \\  
    13&     & 43    & 0 \\  
    14& 76  &       & 28 \\  
    15&     & 71    & 0 \\  
    16& 48  &       & 28 \\
    17&     & 72    & 27 \\  
    18& 120 & 0     &  \\  
    19&     & 27    & 0 \\  
    20& 92  &       & 28 \\  
    21&     & 55    & 0 \\  
    22& 64  &       & 28 \\  
    23&     & 72    & 11 \\  
    24&     & 0     &   \\  
    25&     & 11    & 0 \\  
    26& 36  &       & 28 \\  
    27&     &       & 0 \\  
    28& 8   &       & 28 \\  
    \hline
      & 8   & 11    & 28 \\  
    \hline
\end{tabular}

\subsection*{3. Groups and Rewriting}
% https://www.cs.unm.edu/~mccune/prover9/manual/Dec-2007/output.html
\paragraph{(a)} 
\textit{In mathematics, a group is defined to be a set $G$ with an 
element $I \in G$, a binary operator $∗$ and a unary operator $inv$ satisfying 
four group laws.} \footnote{see the original problem description for a listing
of the four laws.}

\textit{Determine if the following four properties hold, and if not 
determine the size of the smallest finite group for which it does not hold.}

\vspace{3mm}

We use Prover9 and Mace4 to construct our proofs or counter examples. The four
laws are trivially written as \emph{assumptions} in the syntax (taking care
that all elements referenced in the law are universally quantified using the 
\texttt{all} keyword.).

\vspace{3mm}

\begin{equation*}
    I * x = x
\end{equation*}
This property is trivially written as goal in the syntax (once again, taking 
care to universally quantify $x$). Prover9 finds a proof by contradiction: 

\begin{lstlisting}
1  (all x all y all z x * (y * z) = (x * y) * z)    [assumption].§\footnotemark§ 
§\footnotetext{
More on the meaning of the justification can 
be found on the Prover9 website: 
https://www.cs.unm.edu/~mccune/prover9/manual/Dec-2007/output.html under 
`Clause Justification'.}§
2  (all x x * I = x)                                [assumption].
3  (all x x * inv(x) = I)                           [assumption].
4  (all x I * x = x)                                [goal].
8  (x * y) * z = x * (y * z).                       [clausify(1)].
9  x * I = x.                                       [clausify(2)].
10 x * inv(x) = I.                                  [clausify(3)].
11 I * c1 != c1.                                    [deny(4)].
15 x * (I * y) = x * y.                             [para(9(a,1),8(a,1,1)),flip(a)].
16 x * (inv(x) * y) = I * y.                        [para(10(a,1),8(a,1,1)),flip(a)].
21 I * inv(inv(x)) = x.                             [para(10(a,1),16(a,1,2)),rewrite([9(2)]),flip(a)].
23 x * inv(inv(y)) = x * y.                         [para(21(a,1),8(a,2,2)),rewrite([9(2)])].
24 I * x = x.                                       [para(21(a,1),15(a,2)),rewrite([23(5),15(4)])].
25 $F.§\footnotemark§                                              [resolve(24,a,11,a)].
§\footnotetext{\texttt{\$F} means False in Prover9.}§
\end{lstlisting}

\vspace{3mm}

\begin{equation*}
    inv(inv(x)) = x
\end{equation*}
This property is trivially written as goal in the syntax (once again, taking 
care to universally quantify $x$). Prover9 finds a proof by contradiction: 

\begin{lstlisting}
1  (all x all y all z x * (y * z) = (x * y) * z)    [assumption].
2  (all x x * I = x)                                [assumption].
3  (all x x * inv(x) = I)                           [assumption].
5  (all x inv(inv(x)) = x)                          [goal].
8  (x * y) * z = x * (y * z).                       [clausify(1)].
9  x * I = x.                                       [clausify(2)].
10 x * inv(x) = I.                                  [clausify(3)].
12 inv(inv(c2)) != c2.                              [deny(5)].
15 x * (I * y) = x * y.                             [para(9(a,1),8(a,1,1)),flip(a)].
16 x * (inv(x) * y) = I * y.                        [para(10(a,1),8(a,1,1)),flip(a)].
21 I * inv(inv(x)) = x.                             [para(10(a,1),16(a,1,2)),rewrite([9(2)]),flip(a)].
23 x * inv(inv(y)) = x * y.                         [para(21(a,1),8(a,2,2)),rewrite([9(2)])].
24 I * x = x.                                       [para(21(a,1),15(a,2)),rewrite([23(5),15(4)])].
28 x * (inv(x) * y) = y.                            [back_rewrite(16),rewrite([24(5)])].
32 inv(inv(x)) = x.                                 [para(10(a,1),28(a,1,2)),rewrite([9(2)]),flip(a)].
33 $F.                                              [resolve(32,a,12,a)].
\end{lstlisting}

\vspace{3mm}

\begin{equation*}
    inv(x) * x = I
\end{equation*}
This property is trivially written as goal in the syntax (once again, taking 
care to universally quantify $x$). Prover9 finds a proof by contradiction: 

\begin{lstlisting}
1  (all x all y all z x * (y * z) = (x * y) * z)    [assumption].
2  (all x x * I = x)                                [assumption].
3  (all x x * inv(x) = I)                           [assumption].
5  (all x inv(inv(x)) = x)                          [goal].
8  (x * y) * z = x * (y * z).                       [clausify(1)].
9  x * I = x.                                       [clausify(2)].
10 x * inv(x) = I.                                  [clausify(3)].
12 inv(inv(c2)) != c2.                              [deny(5)].
15 x * (I * y) = x * y.                             [para(9(a,1),8(a,1,1)),flip(a)].
16 x * (inv(x) * y) = I * y.                        [para(10(a,1),8(a,1,1)),flip(a)].
21 I * inv(inv(x)) = x.                             [para(10(a,1),16(a,1,2)),rewrite([9(2)]),flip(a)].
23 x * inv(inv(y)) = x * y.                         [para(21(a,1),8(a,2,2)),rewrite([9(2)])].
24 I * x = x.                                       [para(21(a,1),15(a,2)),rewrite([23(5),15(4)])].
28 x * (inv(x) * y) = y.                            [back_rewrite(16),rewrite([24(5)])].
32 inv(inv(x)) = x.                                 [para(10(a,1),28(a,1,2)),rewrite([9(2)]),flip(a)].
33 $F.                                              [resolve(32,a,12,a)].
\end{lstlisting}

\vspace{3mm}

\begin{equation*}
    x * y = y * x
\end{equation*}
This property is trivially written as goal in the syntax (once again, taking 
care to universally quantify $x$ and $y$). Prover9 can't find a proof for this. 
Running the same file through Mace4 gives the counterexample listed below:

\begin{table}[H]  \centering % size 6
G: \{0,1,2,3,4,5\} \hspace{.5cm}
\begin{tabular}{r|rrrrrr}
inv: & 0 & 1 & 2 & 3 & 4 & 5\\
\hline
   & 0 & 1 & 2 & 4 & 3 & 5
\end{tabular} \hspace{.5cm}
\begin{tabular}{rr|rrrrrr}
& & \multicolumn{6}{c}{x} \\  
& *: & 0 & 1 & 2 & 3 & 4 & 5\\
\hline
\multirow{6}{*}{y} & 0 & 0 & 1 & 2 & 3 & 4 & 5 \\
     & 1 & 1 & 0 & 3 & 2 & 5 & 4 \\
     & 2 & 2 & 4 & 0 & 5 & 1 & 3 \\
     & 3 & 3 & 5 & 1 & 4 & 0 & 2 \\
     & 4 & 4 & 2 & 5 & 0 & 3 & 1 \\
     & 5 & 5 & 3 & 4 & 1 & 2 & 0
\end{tabular}
\caption{ }
\end{table} 

\paragraph{(b)} 
\textit{A term rewrite system consists of the single rule
$$a(x,a(y,a(z,u)))\rightarrow a(y,a(z,a(x,u))),$$
in which $a$ is a binary symbol and $x,y,z,u$ are variables.
Moreover, there are constants $b,c,d,e,f,g$. Determine whether
$c$ and $d$ may swapped in $a(b,a(c,a(d,a(e,a(f,a(b,g))))))$ by
rewriting, that is, $a(b,a(c,a(d,a(e,a(f,a(b,g))))))$ rewrites
in a finite number of steps to
$a(b,a(d,a(c,a(e,a(f,a(b,g))))))$.}

\vspace{3mm}

The problem is trivially written in the Prover9 syntax (once again, taking care
of universal quantification of x, y and z) and Prover9 finds the following 
proof by contradiction:

\begin{lstlisting}
1  (all x all y all z all u a(x,a(y,a(z,u))) = a(y,a(z,a(x,u))))        [assumption].
2  a(b,a(c,a(d,a(e,a(f,a(b,g)))))) = a(b,a(d,a(c,a(e,a(f,a(b,g))))))    [goal].
3  a(x,a(y,a(z,u))) = a(z,a(x,a(y,u))).                                 [clausify(1)].
4  a(b,a(d,a(c,a(e,a(f,a(b,g)))))) != a(b,a(c,a(d,a(e,a(f,a(b,g)))))).  [deny(2)].
5  a(x,a(y,a(z,a(u,a(w,v5))))) = a(u,a(x,a(y,a(w,a(z,v5))))).           [para(3(a,1),3(a,1,2,2))].
13 a(b,a(c,a(d,a(e,a(f,a(b,g)))))) != a(b,a(d,a(c,a(b,a(e,a(f,g)))))).[para(3(a,1),4(a,1,2,2,2)),flip(a)].
62 a(x,a(y,a(z,a(u,a(w,a(v5,v6)))))) = a(v5,a(x,a(z,a(y,a(u,a(w,v6)))))).[para(5(a,1),5(a,1,2))].
79 a(d,a(b,a(c,a(e,a(f,a(b,g)))))) != a(b,a(d,a(c,a(b,a(e,a(f,g)))))).  [para(3(a,1),13(a,1))].
80 $F.                                                                  [resolve(79,a,62,a)].
\end{lstlisting}

\subsection*{4. Game of Life}
Game of Life is a mathematics game designed by Conway in the 1970's
\footnote{Scientific American 223. pp. 120–123. ISBN 0-89454-001-7. }. The game
 consists of an $n*m$ grid of `cells', which are 
either alive or dead, and progress through discrete timesteps, or iterations.
Whether a cell is dead or alive in an iteration is based upon the state of its 
eight neighbouring cells in the iteration before. The following four simple rules
decide if a cell is alive in the next iteration:

\begin{enumerate}
    \item A cell dies if less then two of its neighbours are alive
    \item A cell survives if two of its neighbours are alive
    \item A cell is born (or survives) if three of it neighbours are alive
    \item A cell dies if four or more of its neighbours are alive
\end{enumerate}

We aim to investigate different models for the game of life and discuss their
performance. To do so we create a model of the game of life and then use that 
model to decide on a starting configuration which yields at least $16$ living 
cells after $k$ steps on a $5x5$ grid. We run these tests on a laptop with 
2,6Ghz Intel Core i5 processor and 8GB or RAM.

\subsubsection*{Model (a)}
We model the game in YICES using two major sets of variables:
\begin{itemize}
    \item whether a cell is dead or alive in a
        certain iteration, described by an Integer variable which is either 
        $1$ (alive) or $0$ (dead).
    \item how much neighbours of a cell were alive in the previous iteration.
\end{itemize}

For this we introduce $n*m*k + n*m*(k-1)$ variables. 
$c_{xyi}$ for the status of a cell
at position $(x,y)$ at iteration $i$ -with $1 \leq x \leq n$ and $1 \leq y 
\leq m$ and $1 \leq i \leq k$- and $n_{xyi}$ -with $1 \leq x \leq n$ and 
$1 \leq y \leq m$ and $2 \leq i \leq k$-for the number of neighbours
alive at iteration $i-1$. 

We call $C$ the set of all cells, that is 
$\{(x,y) | 1 \leq x \leq n, 1 \leq y \leq m\}$ and $S$ the
set of all steps, that is $\{0 \ldots k\}$. $nb((x,y))$ is a function that 
yields a set of the neighbours of a specific cell at position $(x,y)$.\\  

All cell values should always be either $1$ or $0$:
\begin{equation*}
    \bigwedge_{(x,y) \in C} \bigwedge_{i \in S} c_{xyi} = 0 \vee c_{xyi} = 1
\end{equation*}

\vspace{3mm}

The number of alive neighbours needs to be set for each cell at each timestep:
\begin{equation}
    \bigwedge_{(x,y) \in C} \bigwedge_{i \in S \setminus \{1\}} 
        n_{xyi} = \sum_{(x',y') \in nb((x,y))} c_{x'y'(i-1)}
\end{equation}

\vspace{3mm}

Depending on the number of alive neighbours in the previous iteration the
cells state needs to be updated:
\begin{equation}
\begin{aligned}
    \bigwedge_{(x,y) \in C} \bigwedge_{i \in S \setminus \{1\}} 
        \mathtt{case~} n_{xyi} &\mathtt{~of }\\   
        &< 2 \to c_{xyi} = 0\\  
        &= 2 \to c_{xyi} = c_{xy(i-1)}\\  
        &= 3 \to c_{xyi} = 1\\  
        &\geq 4 \to c_{xyi} = 0
\end{aligned}
\end{equation}

\vspace{3mm}

Finally the final condition, at least 16 cells alive, should be met at the 
last timestep (which we'll number $k$)

\begin{equation*}
	\left( \sum_{(x,y) \in C} c_{xyk} \right) \geq 16
\end{equation*}

\vspace{3mm}

The complete model is the conjunction of the above formulas.

\paragraph{Benchmark} Figure~\ref{fig:modart} shows the running time of model 
(a) in seconds depending on the number of steps asked. It is obvious that the 
model performs poorly and it appears to have a time complexity of $O(k^2)$ 
in regard to the number of steps: $k$.

We've only tested this model up seven steps, as the computation for eight steps
did not finish within 14 minutes. 

\begin{figure}
\includegraphics[width=0.6\textwidth]{encodinga.png}
\caption{Running time of model (a)}
\label{fig:modart}
\end{figure}

\subsubsection*{Model (b)}
We try and improve this result by improving the model. We almost half the
number of variables needed by removing the need for the $n_{xyi}$ variables.\\  

To do this we remove equation (11) and replace equation (12) with (13) which is
(12) with incorporation of the calculation of the number of alive neighbours.

\begin{equation}
\begin{aligned}
    \bigwedge_{(x,y) \in C} \bigwedge_{i \in S \setminus \{1\}} 
        \mathtt{case~} \sum_{(x',y') \in nb((x,y))} c_{x'y'(i-1)} &\mathtt{~of }\\   
        &< 2 \to c_{xyi} = 0\\  
        &= 2 \to c_{xyi} = c_{xy(i-1)}\\  
        &= 3 \to c_{xyi} = 1\\  
        &\geq 4 \to c_{xyi} = 0
\end{aligned}
\end{equation}

\paragraph{Benchmark} Figure~\ref{fig:modbrt} shows the performance of the new
model. The performance has increased enormously, allowing the calculation of 
80 steps in just over 2 minutes. The model appears to have a time complexity of
$O(k)$ in regard to the number of steps: $k$.

\begin{figure}
\includegraphics[width=0.6\textwidth]{encodingb.png}
\caption{Running time of model (b)}
\label{fig:modbrt}
\end{figure}

\subsubsection*{Conclusion}
These two models appear to indicate a huge difference in performance of YICES
depending on the number of variables. Even though the $n_{xyi}$ variables 
\emph{only} serve as helper variables, which would be expected to be easily 
calculated, they have very
significant impact on the running time of the model.
Figure~\ref{fig:comparison} highlights this vast difference in performance.

\begin{figure}
\includegraphics[width=0.6\textwidth]{both.png}
\caption{Comparison between model (a) and model (b)}
\label{fig:comparison}
\end{figure}

\subsubsection*{Discussion}

This experiment just shows the impact of the decrease in the number of 
(superfluous)
variables for this particular problem, with this particular modeling strategy 
and is therefore rather anecdotal. It would be interesting to do experiments 
with other models to be able to have conclusions based on more sound data.\\  

Even though the author has tried as much as possible to ensure equal 
circumstances for each test some tests might have suffered from other programs
running on the test computer. Some measures taken were running a 
minimum on the computer during testing (the OS + Services and Terminal running
YICES and `time') and measuring CPU time used by the YICES process, instead of
real time. The author is therefore confident that the results of these test are
accurate enough for this purpose.\\  

This experiment also only considers the impact of a change in the number of
steps in the model. We do not test the impact of increasing the size of the
grid. However, since the number of variables created in the model is linear with
regards to $n$, $m$ and $k$ similar results would be expected when increasing
$n$ or $m$. However when doing so one would have to take care to increase the
goal ($16$ in our tests) in an appropriate manner, which might not be 
trivial.

Due to time constraints this has not been tested so far, but might be interesting
for future experiments.

\end{document}
