\documentclass[12pt]{article}
\usepackage{a4wide}
\usepackage{latexsym}
\usepackage{amssymb}
\usepackage{epic}
\usepackage{graphicx}
\usepackage[shortlabels]{enumitem}
%\pagestyle{empty}
\newcommand{\tr}{\mbox{\sf true}}
\newcommand{\fa}{\mbox{\sf false}}
\newcommand{\bimp}{\leftrightarrow}


\begin{document}

\section*{Automated Reasoning - Assignment series 1 }

\begin{center}
P.T. Jager, BSc. \\
Radboud universiteit Nijmegen\\
email: {\tt p.jager@student.ru.nl}
\end{center}

\subsection*{Problem 1: Magic factory}

Six trucks have to deliver various obscure building blocks to a special 
factory. There are five types of building blocks:
\begin{enumerate}[(a)]
	\item Nuzzles - 4 pallets - 700 kg 
	\item Skipples - 8 pallets - 1000 kg - \textit{Need to be cooled in
		one of the two cooled trucks}
	\item Crottles - 10 pallets - 1500 kg - \textit{Can not be in the same truck
		as Prittles}
	\item Dupples - 5 pallets - 100 kg - \textit{Not more than 2 pallets per
		truck}
	\item Prittles - $\geq 1$ pallets - 800 kg - \textit{Can not be in same 
		truck as Crottles}
\end{enumerate}
All trucks can cary a maximum of 8 pallets or 7800 kg (whichever is reached 
first.)
What is the maximum number of Prittles that can be delivered and how would the
pallets be divided over the 6 trucks?

For a complete description of the problem please see assignment. 

\subsubsection*{Solution:}

We generalize the solution for any $a$ Nuzzles, $b$ Skipples, $c$ Crottles, 
$d$ Dupples, $e$ Prittles and $n$ trucks.

For doing so we introduce $f*a + f*b + f*c + f*d + f*e$ variables which 
represent each possible pallet in each possible truck: 
\begin{enumerate}[(a)]
	\item $nuzzle_{tn}$ for $t=1,\ldots,f$ and $n=1,\ldots,a$
	\item $skipple_{tn}$ for $t=1,\ldots,f$ and $n=1,\ldots,b$
	\item $crottle_{tn}$ for $t=1,\ldots,f$ and $n=1,\ldots,c$
	\item $dupple_{tn}$ for $t=1,\ldots,f$ and $n=1,\ldots,d$
	\item $prittle_{tn}$ for $t=1,\ldots,f$ and $n=1,\ldots,a$
\end{enumerate}
All of these variables are either $0$ or $1$, when the $nuzzle_{tn}$ 
is $1$ for some $t,n$ then nuzzle $n$ is in truck $t$. Similar for the other 
building blocks.


We generalize this problem to putting $n$ queens on an $n \times n$
chess board, for any $n \geq 1$, with the same restriction that no 
two share a colum, row or diagonal.

For doing so, we introduce $n^2$  
 boolean variables $p_{ij}$ for $i,j = 1,\ldots,n$, where for 
every $i,j = 1,\ldots,n$ the value of $p_{ij}$ will be true if and
only if a queen will be put on position $(i,j)$, that is, in the
$i$-th row and in the $j$-th column.

As we have to put exactly $n$ queens, and no two are allowed to be
on the same row, every row should contain at least one queen. For
row $i$ this is expressed by the formula
\[ \bigvee_{j=1}^n p_{ij}.\]
In a similar way every column should contain at least one queen;
for column $j$ this is expressed by the formula 
\[ \bigvee_{i=1}^n p_{ij}.\]
Next we express that every row contains at most one queen, that
is, for every two distinct positions $j,k$ it is not allowed that
both $p_{ij}$ and $p_{ik}$ are true. For row $i$ this is expressed
by the formula
\[ \bigwedge_{j,k:1 \leq j < k \leq n} \neg p_{ij} \vee \neg p_{ik}.\]
Similarly, every column should contain at most one queen;
for column $j$ this is expressed by the formula 
\[ \bigwedge_{i,k:1 \leq i < k \leq n} \neg p_{ij} \vee \neg p_{kj} \]
Finally, we consider the requirements on diagonals. Two positions
$(i,j)$ and $(k,m)$ are on the same diagonal in the one direction
if and only if $i+k = j+m$, and they are on the same diagonal in 
the other direction if and only if $i-k = j-m$. For every pair of
such positions it is not allowed that they are both occupied by a
queen, so we require
\[ \neg p_{ij} \vee \neg p_{km}.\]
The total formula now consists of the conjunction of all these
ingredients, that is, 
\[ \bigwedge_{i=1}^n (\bigvee_{j=1}^n p_{ij}) \;\; \wedge \]
\[ \bigwedge_{j=1}^n (\bigvee_{i=1}^n p_{ij}) \;\; \wedge \]
\[  \bigwedge_{i=1}^n (\bigwedge_{j,k:1 \leq j < k \leq n} \neg 
p_{ij} \vee \neg p_{ik}) \;\; \wedge \]
\[ \bigwedge_{j=1}^n ( \bigwedge_{i,k:1 \leq i < k \leq n} \neg p_{ij}
\vee \neg p_{kj}) \;\; \wedge \]
\[ \bigwedge_{i,k:1 \leq i < k \leq n} ( \bigwedge_{j,m: i+k = j+m \vee
i-k = j-m} \neg p_{ij} \vee \neg p_{km}) \]

This formula is easily expressed in SMT syntax, for instance, for
$n=8$ one can generate

{\footnotesize

{\tt (benchmark test.smt}

{\tt :extrapreds (}

{\tt (p11) (p12) (p13) (p14) (p15) (p16) (p17) (p18) }

{\tt (p21) (p22) (p23) (p24) (p25) (p26) (p27) (p28) }

{\tt (p31) (p32) (p33) (p34) (p35) (p36) (p37) (p38) }

{\tt (p41) (p42) (p43) (p44) (p45) (p46) (p47) (p48) }

{\tt (p51) (p52) (p53) (p54) (p55) (p56) (p57) (p58) }

{\tt (p61) (p62) (p63) (p64) (p65) (p66) (p67) (p68) }

{\tt (p71) (p72) (p73) (p74) (p75) (p76) (p77) (p78) }

{\tt (p81) (p82) (p83) (p84) (p85) (p86) (p87) (p88) }

{\tt )}

{\tt :formula}

{\tt   (and}

{\tt (or (p11) (p12) (p13) (p14) (p15) (p16) (p17) (p18) )}

{\tt (or (p21) (p22) (p23) (p24) (p25) (p26) (p27) (p28) )}

$\cdots \cdots$

{\tt (or (not p11) (not p12)) }

{\tt (or (not p11) (not p13)) }

$\cdots \cdots$

{\tt )) }
}

Applying {\tt yices -e -smt test.smt} yields the following result
within a fraction of a second:

{\footnotesize

{\tt sat }

{\tt (= p11 false)}

{\tt (= p12 false)}

{\tt (= p13 false)}

{\tt (= p14 true)}

{\tt (= p15 false)}

{\tt (= p16 false)}

{\tt (= p17 false)}

{\tt (= p18 false)}

{\tt (= p21 false)}

{\tt (= p22 true)}

{\tt (= p23 false)}

$\cdots \cdots$ }.

The values that are are true are $p_{14}, p_{22}, p_{37}, p_{43},
p_{56}, p_{68}, p_{75}, p_{81}$. Expressed in a picture this
yields

\vspace{3mm}



\vspace{3mm}

We check that indeed there are eight queens for which no two are
on the same row, column or diagonal.

\vspace{3mm}

{\bf Remark:} 

Our formula contains some redundancy: the requirement that every
row contains exactly one queen implies that there are exactly $n$
queens in total. By expressing that every column contains at least
one queen, from this property one concludes that also every column
contains at most one queen. We chose for this redundancy for
keeping the symmetry in the solution, and also following the
general strategy that introducing redundancy is often good for
efficiency.

\vspace{3mm}

{\bf Generalization:} 

As we generalized our approach for $n$ rather than 8, it is
interesting to see what happens for other values of $n$. For $n
> 10$ we have to take care of the notation: if we keep the
notation then it is not clear whether {\tt p111} represents 
$p_{1,11}$ or $p_{11,1}$. This is solved by putting an extra 
symbol between the two numbers. 

For $n=3$ the resulting formula is unsatisfiable, showing that
there is no solution. For $n = 4,5,6,\ldots$ the formula is
satisfiable, by which is a solution of the problem is found.
Efficiency is not a problem: for $n = 60$ there are 3600
variables and the formula consists of over 350,000 clauses, but 
still {\tt yices} succeeds in finding a solution within a few
seconds.
\end{document}
